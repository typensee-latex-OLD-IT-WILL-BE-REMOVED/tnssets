%\documentclass[12pt,a4paper]{scrartcl}
\documentclass[12pt,a4paper]{article}

\makeatletter % Technical doc - START

\usepackage[utf8]{inputenc}
\usepackage[T1]{fontenc}
\usepackage{ucs}

\usepackage[french]{babel,varioref}

\usepackage[top=2cm, bottom=2cm, left=1.5cm, right=1.5cm]{geometry}
\usepackage{enumitem}

\usepackage{pgffor}

\usepackage{multicol}

\usepackage{makecell}

\usepackage{color}
\usepackage{hyperref}
\hypersetup{
    colorlinks,
    citecolor=black,
    filecolor=black,
    linkcolor=black,
    urlcolor=black
}

\usepackage{amsthm}

\usepackage{tcolorbox}
\tcbuselibrary{listingsutf8}

\usepackage{ifplatform}

\usepackage{ifthen}

\usepackage{macroenvsign}


% Sections numbering

%\renewcommand\thechapter{\Alph{chapter}.}
\renewcommand\thesection{\Roman{section}.}
\renewcommand\thesubsection{\arabic{subsection}.}
\renewcommand\thesubsubsection{\roman{subsubsection}.}



% MISC

\newtcblisting{latexex}{%
	sharp corners,%
	left=1mm, right=1mm,%
	bottom=1mm, top=1mm,%
	colupper=red!75!blue,%
	listing side text
}

\newtcbinputlisting{\inputlatexex}[2][]{%
	listing file={#2},%
	sharp corners,%
	left=1mm, right=1mm,%
	bottom=1mm, top=1mm,%
	colupper=red!75!blue,%
	listing side text
}


\newtcblisting{latexex-flat}{%
	sharp corners,%
	left=1mm, right=1mm,%
	bottom=1mm, top=1mm,%
	colupper=red!75!blue,%
}

\newtcbinputlisting{\inputlatexexflat}[2][]{%
	listing file={#2},%
	sharp corners,%
	left=1mm, right=1mm,%
	bottom=1mm, top=1mm,%
	colupper=red!75!blue,%
}


\newtcblisting{latexex-alone}{%
	sharp corners,%
	left=1mm, right=1mm,%
	bottom=1mm, top=1mm,%
	colupper=red!75!blue,%
	listing only
}

\newtcbinputlisting{\inputlatexexalone}[2][]{%
	listing file={#2},%
	sharp corners,%
	left=1mm, right=1mm,%
	bottom=1mm, top=1mm,%
	colupper=red!75!blue,%
	listing only
}


\newcommand\inputlatexexcodeafter[1]{%
	\begin{center}
		\input{#1}
	\end{center}

	\vspace{-.5em}
	
	Le rendu précédent a été obtenu via le code suivant.
	
	\inputlatexexalone{#1}
}


\newcommand\inputlatexexcodebefore[1]{%
	\inputlatexexalone{#1}
	\vspace{-.75em}
	\begin{center}
		\textit{\footnotesize Rendu du code précédent}
		
		\medskip
		
		\input{#1}
	\end{center}
}


\newcommand\env[1]{\texttt{#1}}
\newcommand\macro[1]{\env{\textbackslash{}#1}}



\setlength{\parindent}{0cm}
\setlist{noitemsep}

\theoremstyle{definition}
\newtheorem*{remark}{Remarque}

\usepackage[raggedright]{titlesec}

\titleformat{\paragraph}[hang]{\normalfont\normalsize\bfseries}{\theparagraph}{1em}{}
\titlespacing*{\paragraph}{0pt}{3.25ex plus 1ex minus .2ex}{0.5em}


\newcommand\separation{
	\medskip
	\hfill\rule{0.5\textwidth}{0.75pt}\hfill
	\medskip
}


\newcommand\extraspace{
	\vspace{0.25em}
}


\newcommand\whyprefix[2]{%
	\textbf{\prefix{#1}}-#2%
}

\newcommand\mwhyprefix[2]{%
	\texttt{#1 = #1-#2}%
}

\newcommand\prefix[1]{%
	\texttt{#1}%
}


\newcommand\inenglish{\@ifstar{\@inenglish@star}{\@inenglish@no@star}}

\newcommand\@inenglish@star[1]{%
	\emph{\og #1 \fg}%
}

\newcommand\@inenglish@no@star[1]{%
	\@inenglish@star{#1} en anglais%
}


\newcommand\ascii{\texttt{ASCII}}


% Example
\newcounter{paraexample}[subsubsection]

\newcommand\@newexample@abstract[2]{%
	\paragraph{%
		#1%
		\if\relax\detokenize{#2}\relax\else {} -- #2\fi%
	}%
}



\newcommand\newparaexample{\@ifstar{\@newparaexample@star}{\@newparaexample@no@star}}

\newcommand\@newparaexample@no@star[1]{%
	\refstepcounter{paraexample}%
	\@newexample@abstract{Exemple \theparaexample}{#1}%
}

\newcommand\@newparaexample@star[1]{%
	\@newexample@abstract{Exemple}{#1}%
}


% Change log
\newcommand\topic{\@ifstar{\@topic@star}{\@topic@no@star}}

\newcommand\@topic@no@star[1]{%
	\textbf{\textsc{#1}.}%
}

\newcommand\@topic@star[1]{%
	\textbf{\textsc{#1} :}%
}

\makeatother % Technical doc - END


\usepackage{tnssets}


\begin{document}

\renewcommand\labelitemi{\raisebox{0.125em}{\tiny\textbullet}}
\renewcommand{\labelitemii}{---}

\title{  %
	Le package \texttt{tnssets}:\\%
	théorie générale des ensembles\\%
	et des applications\\%
	{\footnotesize Code source disponible sur \url{https://github.com/typensee-latex/tnssets.git}.}\\%
{\footnotesize Version \texttt{0.3.0-beta} développée et testée sur \macosxname{}.}%
}
\author{Christophe BAL}
\date{2020-08-05}

\maketitle


\vspace{2em}

\hrule

\tableofcontents

\vspace{1.5em}

\hrule

\newpage

\section{Introduction}

Le package \verb+tnssets+ propose des macros utiles pour la rédaction de texte sur la théorie basique et générale des ensembles et des applications via un codage sémantique simple.


% tnscom used - START
\section{Beta-dépendance}

\verb#tnscom# qui est disponible sur \url{https://github.com/typensee-latex/tnscom.git} est un package utilisé en coulisse.
% tnscom used - END
% List of packages - START
\section{Packages utilisés}

La roue ayant déjà été inventée, le package \verb#tnssets# réutilise les packages suivants sans aucun scrupule.

\begin{multicols}{4}
    \begin{itemize}
        \item \verb#amssymb#
        \item \verb#dsfont#
        \item \verb#forloop#
        \item \verb#mathrsfs#
        \item \verb#mathtools#
        \item \verb#relsize#
        \item \verb#stackengine#
        \item \verb#stmaryrd#
        \item \verb#xstring#
        \item \verb#yhmath#
    \end{itemize}
\end{multicols}
% List of packages - END
\section{Ensembles}

\subsection{Ensembles versus accolades}

\newparaexample{}

\begin{latexex}
$\setgene{1 ; 3 ; 5}$ .
\end{latexex}


% ---------------------- %


\newparaexample{}

Dans l'exemple suivant on utilise l'option \prefix{sb} pour \whyprefix{s}{mall} \whyprefix{b}{races} soit \inenglish{petites accolades}.

\begin{latexex}
$\setgene {\dfrac{1}{3} ; \dfrac{5}{7} ; 
           \dfrac{9}{11}}$

$\setgene*{\dfrac{1}{3} ; \dfrac{5}{7} ;
           \dfrac{9}{11}}$
\end{latexex}


% ---------------------- %


\subsection{Ensembles pour la géométrie} \label{tnssets-geo-sets}

\newparaexample{}

\begin{latexex}
$\setgeo{C}$ ,
$\setgeo{D}$ ou
$\setgeo{d}$
\end{latexex}

\begin{remark}
	Pour le moment, il n'est pas possible de taper \verb+$\setgeo{ABC}$+ avec plusieurs lettres.
\end{remark}


% ---------------------- %


\newparaexample{Avec des indices}

\begin{latexex}
$\setgeo*{C}{1}$ ou
$\setgeo*{C}{2}$
\end{latexex}


% ---------------------- %


\subsection{Ensembles probabilistes} \label{tnssets-proba-sets}

\newparaexample{}

\begin{latexex}
$\setproba{E}$ ou
$\setproba{G}$
\end{latexex}

\begin{remark}
	Pour le moment, il n'est pas possible de taper \verb+$\setproba{ABC}$+ avec plusieurs lettres.
\end{remark}


% ---------------------- %


\newparaexample{Avec des indices}

\begin{latexex}
$\setproba*{E}{1}$ ou
$\setproba*{E}{2}$
\end{latexex}


% ---------------------- %


\subsection{Ensembles pour l'algèbre générale} \label{tnssets-algebra-sets}

\newparaexample{}

\begin{latexex}
$\setalge{A}$ ,
$\setalge{K}$ ,
$\setalge{h}$ ou
$\setalge{k}$
\end{latexex}

\begin{remark}
	Pour le moment, il n'est pas possible de taper \verb+$\setalge{ABC}$+ avec plusieurs lettres.
\end{remark}


% ---------------------- %


\newparaexample{Avec des indices}

\begin{latexex}
$\setalge*{k}{1}$ ou $\setalge*{k}{2}$
\end{latexex}


% ---------------------- %


\subsection{Ensembles classiques en mathématiques et en informatique théorique} \label{tnssets-classical-sets}

\subsubsection{La liste complète}

Dans l'exemple suivant,
$\PP$ désigne l'ensemble des nombres premiers,
$\HH$ celui des quaternions,
$\OO$ celui des octonions et
$\FF$ un ensemble de nombres flottants \emph{(notation à préciser suivant le contexte)}.

\begin{latexex}
$\nullset$

$\NN$ , $\ZZ$ , $\PP$

$\DD$ , $\QQ$ , $\RR$ , $\CC$

$\HH$ , $\OO$

$\FF$
\end{latexex}


% ---------------------- %


\subsubsection{Ensembles classiques suffixés}

L'ensemble $\RR$ nous permet de voir tous les cas possibles. 

\begin{latexex}
$\RRn$ , $\RRp$ , $\RRs$ 

$\RRsn$ , $\RRsp$
\end{latexex}


Nous avons utilisé les suffixes \prefix{n} pour \whyprefix{n}{égatif}, \prefix{p} pour \whyprefix{p}{ositif} et \prefix{s} pour \whyprefix{s}{tar} soit \inenglish{étoile}. Il y a aussi les suffixes composites \prefix{sn} et \prefix{sp}.

\medskip

Notez qu'il est interdit d'utiliser \verb+$\CCn$+ pour $\setspecial{\CC}{n}$ car l'ensemble $\CC$ ne possède pas de structure ordonnée standard. Jetez un oeil à la section suivante pour apprendre à taper $\setspecial{\CC}{n}$ si vous en avez besoin. L'interdiction est ici purement sémantique !

\medskip

\begin{remark}
	La table \ref{tnssets-table:suffixes-sets} \vpageref{tnssets-table:suffixes-sets} montre les associations autorisées entre ensembles classiques et suffixes.
\end{remark}

% == Table of suffixes - START == %

\begin{table}[h]
    \caption{Suffixes}
    \begin{center}
        \begin{tabular}{c|c|c|c|c|c}
              & \verb+n+ & \verb+p+ & \verb+s+ & \verb+sn+ & \verb+sp+ \\
            \hline \makecell{\macro{NN}} &          &          & $\times$ &          &          \\
            \hline \makecell{\macro{PP}} &          &          &          &          &          \\
            \hline \makecell{\macro{ZZ}\\\macro{DD}\\\macro{QQ}\\\macro{RR}} & $\times$ & $\times$ & $\times$ & $\times$ & $\times$ \\
            \hline \makecell{\macro{CC}\\\macro{HH}\\\macro{OO}} &          &          & $\times$ &          &          \\
            \hline \makecell{\macro{FF}} & $\times$ & $\times$ & $\times$ & $\times$ & $\times$ \\
        \end{tabular}
    \end{center}
    \label{tnssets-table:suffixes-sets}
\end{table}

% == Table of suffixes - END == %


% ---------------------- %


\subsubsection{Des suffixes à la carte}

Dans l'exemple suivant, il faut savoir que le 2\ieme{} argument ne peut prendre que les valeurs \prefix{n}, \prefix{p}, \prefix{s}, \prefix{sn} ou \prefix{sp}.

\begin{latexex}
$\setspecial{\CC}{n}$ ,
$\setspecial{\HH}{sp}$ ou
$\setspecial*{\setproba{P}}{n}$
\end{latexex}


% ---------------------- %
\section{Intervalles}

\subsection{Intervalles réels - Notation française (?\,)}

\newparaexample{}

Dans cet exemple, la syntaxe fait référence à 
\whyprefix{O}{pened} et \whyprefix{C}{losed}
pour
\inenglish{ouvert et fermé}.
Nous verrons que \prefix{CC} et \prefix{OO} sont contractés en \prefix{C} et \prefix{O}.
Notez au passage que la macro utilisée résout un problème d'espacement vis à vis du signe $=$ .

\begin{latexex}
$I = ]a ; b] = \intervalOC{a}{b}$
\end{latexex}


% ---------------------- %


\newparaexample{}

Les crochets s'étendent verticalement automatiquement. Pour empêcher cela, il suffit d'utiliser la version étoilée de la macro.
Dans ce cas, les crochets restent tout de même un peu plus grands que des crochets utilisés directement. Voici un exemple.

\begin{latexex}
$\displaystyle
 \intervalC{ \frac{1}{2} }{ 1^{2^{3}} }
 =
 [ \frac{1}{2} ; 1^{2^{3}} ]
 =
 \intervalC*{ \frac{1}{2} }{ 1^{2^{3}} }$
\end{latexex}


% ---------------------- %


\subsection{Intervalles réels -- Notation américaine}

Dans l'exemple suivant la syntaxe fait référence à \whyprefix{P}{arenthèse}. Cette notation est utilisée aux États Unis.

\begin{latexex}
$\intervalPC{a}{b} = \intervalOC{a}{b}$
et
$\intervalP{a}{b} = \intervalO{a}{b}$.
\end{latexex}


% ---------------------- %


\subsection{Intervalles discrets d'entiers}


Dans l'exemple suivant la syntaxe fait référence à $\ZZ$ l'ensemble des entiers relatifs.

\begin{latexex}
 $\ZintervalC{-1}{4} =
  \{ -1 ; 0 ; 1 ; 2 ; 3 ; 4 \}$
 
 $\ZintervalC{-1}{4} =
  \ZintervalO{-2}{5}$.
\end{latexex}


% ---------------------- %
\section{Unions et intersections en mode ligne}

\LaTeX{} permet d'afficher sans souci $\cup_{k=1}^{n}$ mais ne propose pas $\dcup_{k=1}^{n}$.
Les macros \macro{dcap}, \macro{dcup} et \macro{dsqcup} donnent accès à ce type de fonctionnalité pour $\cap$ , $\cup$ et $\sqcup$ respectivement.
Voici des exemples d'utilisation.


% ---------------------- %


\newparaexample{Les symboles \og seuls \fg}

\begin{latexex}
$A \dcap B = C \cap D$

$A \dcup B = C \cup D$

$A \dsqcup B = C \sqcup D$
\end{latexex}


% ---------------------- %


\newparaexample{Des intersections indicées}

Ci-dessous est utilisée la macro \macro{bigcap} proposée par le package \verb+amssymb+.

\begin{latexex}
$\cap_{k=1}^{n}    A_k$ ,
$\dcap_{k=1}^{n}   B_k$ ,
$\bigcap_{k=1}^{n} C_k$ ,
$\displaystyle%
 \bigcap_{k=1}^{n} D_k$
\end{latexex}


% ---------------------- %


\newparaexample{Des unions indicées}

Ci-dessous est utilisée la macro \macro{bigcup} proposée par le package \verb+amssymb+.

\begin{latexex}
$\cup_{k=1}^{n}    A_k$ ,
$\dcup_{k=1}^{n}   B_k$ ,
$\bigcup_{k=1}^{n} C_k$ ,
$\displaystyle%
 \bigcup_{k=1}^{n} D_k$
\end{latexex}


% ---------------------- %


\newparaexample{Des unions disjointes indicées}

Ci-dessous sont utilisées les macros \macro{sqcup} et \macro{bigsqcup} proposée par le package \verb+amssymb+.

\begin{latexex}
$\sqcup_{k=1}^{n}    A_k$ ,
$\dsqcup_{k=1}^{n}   B_k$ ,
$\bigsqcup_{k=1}^{n} C_k$ ,
$\displaystyle%
 \bigsqcup_{k=1}^{n} D_k$
\end{latexex}


% ---------------------- %
\section{Des applications très spéciales}

\subsection{Fonction identité}

\begin{latexex}
$\id_A$ vérifie par définition
$\forall x \in A$, $\id(x) = x$.
\end{latexex}


% ---------------------- %


\subsection{Fonction caractéristique}

\newparaexample{Version très utilisée}

\begin{latexex}
$\caract_A$ vérifie par définition
$\forall x \in A$, $\caract_A(x) = 1$ et
$\forall x \notin A$, $\caract_A(x) = 0$.
\end{latexex}


% ---------------------- %


\newparaexample{Version alternative}

\begin{latexex}
$\caractone_A = \caract_A$
\end{latexex}


% ---------------------- %
\section{Applications}

\subsection{Cardinal, image et compagnie}

\newparaexample{Cardinal}

\begin{latexex}
$\card* E = \card E$
\end{latexex}


% ---------------------- %


\newparaexample{Image et compagnie}

Ci-dessous se trouve la macro \macro{ker} proposée par \verb+amsmath+ qui est importé par \verb+tnssets+.

\begin{latexex}
$\ker f$ , $\dom f$ ,
$\im f$ ou $\codom f$
\end{latexex}


% ---------------------- %
%\section{Applications}

\subsection{Application totale, partielle, injective, surjective et/ou bijective}

Voici des symboles qui, bien que très techniques, facilitent la rédaction de documents à propos des applications totales ou partielles
\footnote{
	$a: E \to F$ est une application totale si $\forall x \in E$, $\exists\,! y \in F$ tel que $y = a(x)$. 
	Plus généralement, $f: E \to F$ est une application partielle si $\forall x \in E$, $\exists_{\leq 1} y \in F$ tel que $y = f(x)$, autrement dit soit $f(x)$ existe dans $F$, soit $f$ n'est pas définie en $x$.
}
\emph{(on parle aussi d'applications, sans qualificatif, et de fonctions)}.


% ---------------------- %


\newparaexample{Applications totales}

\begin{latexex-flat}
$f: A \to B$ est une application totale, c'est à dire définie sur $A$ tout entier.

$i: C \onetoone D$ est une application totale injective.

$s: E \onto F$ est une application totale surjective.

$b: G \biject H$ est une application totale bijective.
\end{latexex-flat}


% ---------------------- %


\newparaexample{Applications partielles}

\begin{latexex-flat}
$f: A \pto B$ est une application partielle, c'est à dire définie sur
un sous-ensemble de $A$.

$i: C \ponetoone D$ est une application partielle injective.

$s: E \ponto F$ est une application partielle surjective.

$b: G \pbiject H$ est une application partielle bijective.
\end{latexex-flat}


% ---------------------- %
%\section{Applications}

\subsection{Définition explicite d'une fonction}

\newparaexample{Écriture par défaut}

\begin{latexex}
$\funcdef{f}{x}{x^2}%
            {I}{J}$
\end{latexex}


\begin{remark}
	Même si cela est peu utile, vous pouvez utiliser la mise en forme dans du texte pour obtenir
	$\funcdef{f}{x}{x^2}%
            {I}{J}$
    mais c'est un peu affreux.
\end{remark}

% ---------------------- %


\newparaexample{Écriture alternative}

On peut cacher le trait vertical via l'option \prefix{s} pour \whyprefix{s}{hort} soit \inenglish{court}.

\begin{latexex}
$\funcdef[s]{f}{x}{x^2}%
               {I}{J}$
\end{latexex}


% ---------------------- %


\newparaexample{Écriture en ligne}

Pour avoir tout sur une ligne, ce qui est l'idéal pour une insertion dans du texte, il suffit d'utiliser l'option \prefix{h} pour \whyprefix{h}{orizontal}.
\begin{latexex}
Soit $\funcdef[h]{f}{x}{x^2}%
                    {I}{J}$ ...
\end{latexex}


% ---------------------- %


\newparaexample{Écriture en ligne incomplète}

En mode horizontal, les ensembles peuvent être de valeur vide pour ne pas les indiquer.

\begin{latexex}
$\funcdef[h]{f}{x}{x^2}{}{J}$

$\funcdef[h]{f}{x}{x^2}{I}{}$

$\funcdef[h]{f}{x}{x^2}{}{}$
\end{latexex}


% ---------------------- %


\newparaexample{Écriture textuelle}

On peut enfin obtenir une version \emph{\og textuelle \fg} via la macro \macro{txtfuncdef} où \prefix{txt} est pour \prefix{texte}.
Cette macro  ne s'utilise pas en mode mathématique et elle accepte l'omission des ensembles.

\begin{latexex}
\txtfuncdef{f}{x}{x^2}{I}{J}

\txtfuncdef{f}{x}{x^2}{}{J}

\txtfuncdef{f}{x}{x^2}{I}{}

\txtfuncdef{f}{x}{x^2}{}{}
\end{latexex}



% ---------------------- %
%\section{Applications}

\subsection{Composition}

\newparaexample{Opérateur}

La macro \macro{compo} est juste une version un peu plus petite de \macro{circ}.

\begin{latexex}
$f \compo g$ et non
$f \circ g$
\end{latexex}


% ---------------------- %


\newparaexample{Compositions successives}

La macro \macro{multicompo} sert à indiquer la composition d'une application plusieurs fois de suite par elle-même
\footnote{
	Une telle fonction $f$ doit vérifier $\im f \subseteq \dom f$.
}.
Voici toutes les mises en forme disponibles où l'option \verb#exp# nécessite que le nombre d'applications composées soit un naturel non nul connu.
Vous noterez que l'écriture par défaut, qui n'est pas standard, n'est pas $f^{(p)}$ car cette notation est traditionnellement utilisée pour indiquer la dérivée p\ieme{} d'une application.

\begin{latexex}
 $\multicompo     {g}{5}
= \multicompo[exp]{g}{5}$

 $\multicompo     {f}{p}
= \multicompo[dot]{f}{p}$
\end{latexex}


\begin{remark}
	La convention retenue est analogue à ce que l'on fait avec les puissances de nombres réels.
	En particulier, $\multicompo{f}{1} = \multicompo[exp]{f}{1}$ et $\multicompo{f}{0} = \id_{\dom f}$ .
\end{remark}


% ---------------------- %
\newpage

\section{Historique}

Nous ne donnons ici qu'un très bref historique récent
\footnote{
	On ne va pas au-delà de un an depuis la dernière version.
}
de \verb+tnssets+ à destination de l'utilisateur principalement.
Tous les changements sont disponibles uniquement en anglais dans le dossier \verb+change-log+ : voir le code source de \verb+tnssets+ sur \verb+github+.

\begin{description}
% Changes shown - START

    \medskip
    \item[2020-08-05] Nouvelle version mineure \verb+0.3.0-beta+.
    
    \begin{itemize}[itemsep=.5em]
        \item \topic*{Définition explicite d'une fonction}
              intégration de deux macros proposées avant par \verb#tnsana# disponible sur \url{https://github.com/typensee-latex/tnsana.git}.
    
        \begin{itemize}[itemsep=.5em]
            \item \macro{funcdef} peut produire trois versions symboliques.
    
            \item \macro{txtfuncdef} produit une version textuelle courte.
        \end{itemize}
    
    
        \item \topic{Fonctions spéciales}
    
        \begin{itemize}[itemsep=.5em]
            \item Ajout de \macro{id} pour la fonction identité.
    
            \item Ajout de \macro{caract} et \macro{caractone} pour deux versions de la fonction caratéristque d'un ensemble.
        \end{itemize}
    \end{itemize}
    
    \separation
% ------------------------ %

    \medskip
    \item[2020-07-30] Nouvelle version mineure \verb+0.2.0-beta+.
    
    \begin{itemize}[itemsep=.5em]
        \item \topic{Composition d'applications}
        \begin{itemize}[itemsep=.5em]
            \item \macro{compo} est un opérateur de composition de deux applications.
    
            \item \macro{multicomp} permet d'indiquer des compositions successives d'une application par elle-même.
        \end{itemize}
    \end{itemize}
    
    \separation

% ------------------------ %

    \medskip
    \item[2020-07-10] Première version \verb+0.0.0-beta+.
% ------------------------ %

% Changes shown - END 
\end{description}


\newpage
\section{Toutes les fiches techniques} \label{techincal-ids}









\subsection{Ensembles}

\subsubsection{Ensembles versus accolades}



\IDmacro{setgene}{1}{1}

\IDoption{} la valeur par défaut est \verb+b+.  Voici les différentes valeurs possibles.
\begin{enumerate}
	\item \verb+b+ : on utilise des accolades extensibles.

	\item \verb+sb+ : on utilise des accolades non extensibles.
\end{enumerate}

\IDarg{} la définition de l'ensemble.

\IDarg{} la définition de l'ensemble.


% ---------------------- %


\subsubsection{Ensembles pour la géométrie} 



\IDmacro[a]{setgeo}{1}

\IDarg{} un seul caractère \ascii{} indiquant un ensemble géométrique.


\separation


\IDmacro[a]{setgeo*}{2}

\IDarg{1} un seul caractère \ascii{} indiquant $\setgeo{U}$ dans le nom $\setgeo*{U}{d}$ d'un ensemble géométrique.

\IDarg{2} un texte donnant $d$ dans le nom $\setgeo*{U}{d}$ d'un ensemble géométrique.


% ---------------------- %


\subsubsection{Ensembles probabilistes} 



\IDmacro[a]{setproba}{1}

\IDarg{} un seul caractère \ascii{} majuscule indiquant un ensemble probabiliste.


\separation


\IDmacro[a]{setproba*}{2}

\IDarg{1} un seul caractère \ascii{} majuscule indiquant $\setproba{U}$ dans le nom $\setproba*{U}{d}$ d'un ensemble probabiliste.

\IDarg{2} un texte donnant $d$ dans le nom $\setproba*{U}{d}$ d'un ensemble probabiliste.


% ---------------------- %


\subsubsection{Ensembles pour l'algèbre générale} 



\IDmacro[a]{setalge}{1}

\IDarg{} soit l'une des lettres  \texttt{h} et \texttt{k}, soit un seul caractère \ascii{} majuscule indiquant un ensemble de type anneau ou corps.


\separation


\IDmacro[a]{setalge*}{2}

\IDarg{1} un seul caractère \ascii{} indiquant $\setalge{U}$ dans le nom $\setalge*{U}{d}$ d'un ensemble de type anneau ou corps.

\IDarg{2} un texte donnant $d$ dans le nom $\setalge*{U}{d}$ d'un ensemble de type anneau ou corps.


% ---------------------- %


\subsubsection{Ensembles classiques en mathématiques et en informatique théorique} 

\paragraph{Ensembles classiques suffixés}



% == Docs for classical sets - START == %
\foreach \k in {NN, NNs}{
    \IDope{\k} \quad
}
                
\separation

\foreach \k in {PP}{
    \IDope{\k} \quad
}
                
\separation

\foreach \k in {ZZ, ZZn, ZZp, ZZs, ZZsn, ZZsp}{
    \IDope{\k} \quad
}
                
\separation

\foreach \k in {DD, DDn, DDp, DDs, DDsn, DDsp}{
    \IDope{\k} \quad
}
                
\separation

\foreach \k in {QQ, QQn, QQp, QQs, QQsn, QQsp}{
    \IDope{\k} \quad
}
                
\separation

\foreach \k in {RR, RRn, RRp, RRs, RRsn, RRsp}{
    \IDope{\k} \quad
}
                
\separation

\foreach \k in {CC, CCs}{
    \IDope{\k} \quad
}
                
\separation

\foreach \k in {HH, HHs}{
    \IDope{\k} \quad
}
                
\separation

\foreach \k in {OO, OOs}{
    \IDope{\k} \quad
}
                % == Docs for classical sets - END == %


% ---------------------- %


\subsubsection{Ensembles classiques en mathématiques et en informatique théorique} 

\paragraph{Des suffixes à la carte}



\IDmacro[a]{setspecial }{2}

\IDmacro[a]{setspecial*}{2}

\IDarg{1} l'ensemble à "suffixer".

\IDarg{2} l'un des suffixes \prefix{n}, \prefix{p}, \prefix{s}, \prefix{sn} ou \prefix{sp}.


\subsection{Intervalles}

\subsubsection{Intervalles réels - Notation française (?\,)}



Pour toutes les macros ci-dessous, la version non étoilée produit des délimiteurs qui s'étirent si besoin verticalement, tandis que la version étoilée ne le fait pas.


\separation


% Docs for french real intervals - START

\IDmacro[a]{intervalCO }{2}

\IDmacro[a]{intervalCO*}{2}

\IDarg{1} borne inférieure $a$ de l'intervalle $\intervalCO{a}{b}$.

\IDarg{2} borne supérieure $b$ de l'intervalle $\intervalCO{a}{b}$.


\separation


\IDmacro[a]{intervalC }{2}

\IDmacro[a]{intervalC*}{2}

\IDarg{1} borne inférieure $a$ de l'intervalle $\intervalC{a}{b}$.

\IDarg{2} borne supérieure $b$ de l'intervalle $\intervalC{a}{b}$.


\separation


\IDmacro[a]{intervalO }{2}

\IDmacro[a]{intervalO*}{2}

\IDarg{1} borne inférieure $a$ de l'intervalle $\intervalO{a}{b}$.

\IDarg{2} borne supérieure $b$ de l'intervalle $\intervalO{a}{b}$.


\separation


\IDmacro[a]{intervalOC }{2}

\IDmacro[a]{intervalOC*}{2}

\IDarg{1} borne inférieure $a$ de l'intervalle $\intervalOC{a}{b}$.

\IDarg{2} borne supérieure $b$ de l'intervalle $\intervalOC{a}{b}$.

% Docs for french real intervals - END


% ---------------------- %


\subsubsection{Intervalles réels -- Notation américaine}



Pour toutes les macros ci-dessous, la version non étoilée produit des délimiteurs qui s'étirent si besoin verticalement, tandis que la version étoilée ne le fait pas.


\separation


% Docs for american real intervals - START

\IDmacro[a]{intervalCP }{2}

\IDmacro[a]{intervalCP*}{2}

\IDarg{1} borne inférieure $a$ de l'intervalle $\intervalCP{a}{b}$.

\IDarg{2} borne supérieure $b$ de l'intervalle $\intervalCP{a}{b}$.


\separation


\IDmacro[a]{intervalP }{2}

\IDmacro[a]{intervalP*}{2}

\IDarg{1} borne inférieure $a$ de l'intervalle $\intervalP{a}{b}$.

\IDarg{2} borne supérieure $b$ de l'intervalle $\intervalP{a}{b}$.


\separation


\IDmacro[a]{intervalPC }{2}

\IDmacro[a]{intervalPC*}{2}

\IDarg{1} borne inférieure $a$ de l'intervalle $\intervalPC{a}{b}$.

\IDarg{2} borne supérieure $b$ de l'intervalle $\intervalPC{a}{b}$.

% Docs for american real intervals - END


% ---------------------- %


\subsubsection{Intervalles discrets d'entiers}



Pour toutes les macros ci-dessous, la version non étoilée produit des délimiteurs qui s'étirent si besoin verticalement, tandis que la version étoilée ne le fait pas.


\separation


% Docs for discrete intervals - START

\IDmacro[a]{ZintervalCO }{2}

\IDmacro[a]{ZintervalCO*}{2}

\IDarg{1} borne inférieure $a$ de l'intervalle $\ZintervalCO{a}{b}$.

\IDarg{2} borne supérieure $b$ de l'intervalle $\ZintervalCO{a}{b}$.


\separation


\IDmacro[a]{ZintervalC }{2}

\IDmacro[a]{ZintervalC*}{2}

\IDarg{1} borne inférieure $a$ de l'intervalle $\ZintervalC{a}{b}$.

\IDarg{2} borne supérieure $b$ de l'intervalle $\ZintervalC{a}{b}$.


\separation


\IDmacro[a]{ZintervalO }{2}

\IDmacro[a]{ZintervalO*}{2}

\IDarg{1} borne inférieure $a$ de l'intervalle $\ZintervalO{a}{b}$.

\IDarg{2} borne supérieure $b$ de l'intervalle $\ZintervalO{a}{b}$.


\separation


\IDmacro[a]{ZintervalOC }{2}

\IDmacro[a]{ZintervalOC*}{2}

\IDarg{1} borne inférieure $a$ de l'intervalle $\ZintervalOC{a}{b}$.

\IDarg{2} borne supérieure $b$ de l'intervalle $\ZintervalOC{a}{b}$.

% Docs for discrete intervals - END


\subsection{Unions et intersections en mode ligne}



\IDope{dcap} 

\IDope{dcup}

\IDope{dsqcup}


\subsection{Des applications très spéciales}

\subsubsection{Fonction identité}



\IDope{id}


% ---------------------- %


\subsubsection{Fonction caractéristique}



\IDope{caract}

\IDope{caractone}


\subsection{Applications}

\subsubsection{Cardinal, image et compagnie}



\IDope{card}

\IDope{card*}


\separation


\IDope{dom}

\IDope{codom}

\IDope{im}


\subsubsection{Application totale, partielle, injective, surjective et/ou bijective}



\vspace{-.75em}
\begin{multicols}{2}
    \IDope{to}
    
    \IDope{onetoone}
    
    \IDope{onto}
    
    \IDope{biject}
    
    \IDope{pto}
    
    \IDope{ponetoone}
    
    \IDope{ponto}
    
    \IDope{pbiject}
\end{multicols}


\subsubsection{Définition explicite d'une fonction}



\IDmacro{funcdef}{1}{5}


\IDoption{}  la valeur par défaut \verb+u+. 
\begin{enumerate}
	\item \verb+u+ : écriture empilée avec un trait vertical.
	
	\item \verb+s+ : écriture empilée sans trait vertical.

	\item \verb+h+ : écriture horizontale en ligne.
\end{enumerate}


\IDarg{1} la fonction.

\IDarg{2} la variable.

\IDarg{3} la formule explicite de définition.

\IDarg{4} l'ensemble de départ. Cet argument peut être vide si le mode \verb+h+ est activé

\IDarg{5} l'ensemble d'arrivée.


\separation


\IDmacro[a]{txtfuncdef}{5}


\IDargs{1..5} voir les explications données ci-dessus pour la macro \macro{funcdef}.


\subsubsection{Composition}



\IDope{compo}  \hfill \mwhyprefix{compo}{sition}


\separation

\IDmacro{multicompo}{1}{2}

\IDoption{} la valeur par défaut est \verb+r+. 
            Voici les différentes valeurs possibles.
\begin{enumerate}
	\item \verb+r  + : écriture utilisant des chevrons.  \hfill \mwhyprefix{r}{after}.

	\extraspace 
	
	\item \verb+exp+ : écriture développée.  \hfill \mwhyprefix{exp}{and}.

	\item \verb+dot+ : écriture faussement développée utilisant des points de suspension.
\end{enumerate}

\IDarg{1} l'application.

\IDarg{2} le nombre d'applications composées.




\end{document}
