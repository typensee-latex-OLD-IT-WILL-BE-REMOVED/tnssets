\documentclass[12pt,a4paper]{article}


\makeatletter
    \usepackage[utf8]{inputenc}
\usepackage[T1]{fontenc}
\usepackage{ucs}

\usepackage[french]{babel,varioref}

\usepackage[top=2cm, bottom=2cm, left=1.5cm, right=1.5cm]{geometry}
\usepackage{enumitem}

\usepackage{multicol}

\usepackage{makecell}

\usepackage{color}
\usepackage{hyperref}
\hypersetup{
    colorlinks,
    citecolor=black,
    filecolor=black,
    linkcolor=black,
    urlcolor=black
}

\usepackage{amsthm}

\usepackage{tcolorbox}
\tcbuselibrary{listingsutf8}

\usepackage{ifplatform}

\usepackage{ifthen}

\usepackage{cbdevtool}



% MISC

\newtcblisting{latexex}{%
	sharp corners,%
	left=1mm, right=1mm,%
	bottom=1mm, top=1mm,%
	colupper=red!75!blue,%
	listing side text
}

\newtcbinputlisting{\inputlatexex}[2][]{%
	listing file={#2},%
	sharp corners,%
	left=1mm, right=1mm,%
	bottom=1mm, top=1mm,%
	colupper=red!75!blue,%
	listing side text
}


\newtcblisting{latexex-flat}{%
	sharp corners,%
	left=1mm, right=1mm,%
	bottom=1mm, top=1mm,%
	colupper=red!75!blue,%
}

\newtcbinputlisting{\inputlatexexflat}[2][]{%
	listing file={#2},%
	sharp corners,%
	left=1mm, right=1mm,%
	bottom=1mm, top=1mm,%
	colupper=red!75!blue,%
}


\newtcblisting{latexex-alone}{%
	sharp corners,%
	left=1mm, right=1mm,%
	bottom=1mm, top=1mm,%
	colupper=red!75!blue,%
	listing only
}

\newtcbinputlisting{\inputlatexexalone}[2][]{%
	listing file={#2},%
	sharp corners,%
	left=1mm, right=1mm,%
	bottom=1mm, top=1mm,%
	colupper=red!75!blue,%
	listing only
}


\newcommand\inputlatexexcodeafter[1]{%
	\begin{center}
		\input{#1}
	\end{center}

	\vspace{-.5em}
	
	Le rendu précédent a été obtenu via le code suivant.
	
	\inputlatexexalone{#1}
}


\newcommand\inputlatexexcodebefore[1]{%
	\inputlatexexalone{#1}
	\vspace{-.75em}
	\begin{center}
		\textit{\footnotesize Rendu du code précédent}
		
		\medskip
		
		\input{#1}
	\end{center}
}


\newcommand\env[1]{\texttt{#1}}
\newcommand\macro[1]{\env{\textbackslash{}#1}}



\setlength{\parindent}{0cm}
\setlist{noitemsep}

\theoremstyle{definition}
\newtheorem*{remark}{Remarque}

\usepackage[raggedright]{titlesec}

\titleformat{\paragraph}[hang]{\normalfont\normalsize\bfseries}{\theparagraph}{1em}{}
\titlespacing*{\paragraph}{0pt}{3.25ex plus 1ex minus .2ex}{0.5em}


\newcommand\separation{
	\medskip
	\hfill\rule{0.5\textwidth}{0.75pt}\hfill
	\medskip
}


\newcommand\extraspace{
	\vspace{0.25em}
}


\newcommand\whyprefix[2]{%
	\textbf{\prefix{#1}}-#2%
}

\newcommand\mwhyprefix[2]{%
	\texttt{#1 = #1-#2}%
}

\newcommand\prefix[1]{%
	\texttt{#1}%
}


\newcommand\inenglish{\@ifstar{\@inenglish@star}{\@inenglish@no@star}}

\newcommand\@inenglish@star[1]{%
	\emph{\og #1 \fg}%
}

\newcommand\@inenglish@no@star[1]{%
	\@inenglish@star{#1} en anglais%
}


\newcommand\ascii{\texttt{ASCII}}


% Example
\newcounter{paraexample}[subsubsection]

\newcommand\@newexample@abstract[2]{%
	\paragraph{%
		#1%
		\if\relax\detokenize{#2}\relax\else {} -- #2\fi%
	}%
}



\newcommand\newparaexample{\@ifstar{\@newparaexample@star}{\@newparaexample@no@star}}

\newcommand\@newparaexample@no@star[1]{%
	\refstepcounter{paraexample}%
	\@newexample@abstract{Exemple \theparaexample}{#1}%
}

\newcommand\@newparaexample@star[1]{%
	\@newexample@abstract{Exemple}{#1}%
}


% Change log
\newcommand\topic{\@ifstar{\@topic@star}{\@topic@no@star}}

\newcommand\@topic@no@star[1]{%
	\textbf{\textsc{#1}.}%
}

\newcommand\@topic@star[1]{%
	\textbf{\textsc{#1} :}%
}


	
    \usepackage{03-explicit-def}
\makeatother

   
\begin{document}

%\section{Applications}

\subsection{Définition explicite d'une fonction}

\newparaexample{Écriture par défaut}

\begin{latexex}
$\funcdef{f}{x}{x^2}%
            {I}{J}$
\end{latexex}


\begin{remark}
	Même si cela est peu utile, vous pouvez utiliser la mise en forme dans du texte pour obtenir
	$\funcdef{f}{x}{x^2}%
            {I}{J}$
    mais c'est un peu affreux.
\end{remark}

% ---------------------- %


\newparaexample{Écriture alternative}

On peut cacher le trait vertical via l'option \prefix{s} pour \whyprefix{s}{hort} soit \inenglish{court}.

\begin{latexex}
$\funcdef[s]{f}{x}{x^2}%
               {I}{J}$
\end{latexex}


% ---------------------- %


\newparaexample{Écriture en ligne}

Pour avoir tout sur une ligne, ce qui est l'idéal pour une insertion dans du texte, il suffit d'utiliser l'option \prefix{h} pour \whyprefix{h}{orizontal}.
\begin{latexex}
Soit $\funcdef[h]{f}{x}{x^2}%
                    {I}{J}$ ...
\end{latexex}


% ---------------------- %


\newparaexample{Écriture en ligne incomplète}

En mode horizontal, les ensembles peuvent être de valeur vide pour ne pas les indiquer.

\begin{latexex}
$\funcdef[h]{f}{x}{x^2}{}{J}$

$\funcdef[h]{f}{x}{x^2}{I}{}$

$\funcdef[h]{f}{x}{x^2}{}{}$
\end{latexex}


% ---------------------- %


\newparaexample{Écriture textuelle}

On peut enfin obtenir une version \emph{\og textuelle \fg} via la macro \macro{txtfuncdef} où \prefix{txt} est pour \prefix{texte}.
Cette macro ne s'utilise qu'en mode texte
\footnote{
	Logic ! Isn't it ?
}
et elle accepte l'omission de l'un ou des deux ensembles.

\begin{latexex}
\txtfuncdef{f}{x}{x^2}{I}{J}

\txtfuncdef{f}{x}{x^2}{}{J}

\txtfuncdef{f}{x}{x^2}{I}{}

\txtfuncdef{f}{x}{x^2}{}{}
\end{latexex}



% ---------------------- %


\subsection{Fiches techniques}

\IDmacro{funcdef}{1}{5}


\IDoption{}  la valeur par défaut \verb+u+. 
\begin{enumerate}
	\item \verb+u+ : écriture empilée avec un trait vertical.
	
	\item \verb+s+ : écriture empilée sans trait vertical.

	\item \verb+h+ : écriture horizontale en ligne.
\end{enumerate}


\IDarg{1} la fonction.

\IDarg{2} la variable.

\IDarg{3} la formule explicite de définition.

\IDarg{4} l'ensemble de départ. Cet argument peut être vide si le mode \verb+h+ est activé

\IDarg{5} l'ensemble d'arrivée.


\separation


\IDmacro[a]{txtfuncdef}{5}


\IDargs{1..5} voir les explications données ci-dessus pour la macro \macro{funcdef}.


\end{document}
