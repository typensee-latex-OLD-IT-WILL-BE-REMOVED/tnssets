\documentclass[12pt,a4paper]{article}

\makeatletter
	\usepackage[utf8]{inputenc}
\usepackage[T1]{fontenc}
\usepackage{ucs}

\usepackage[french]{babel,varioref}

\usepackage[top=2cm, bottom=2cm, left=1.5cm, right=1.5cm]{geometry}
\usepackage{enumitem}

\usepackage{pgffor}

\usepackage{multicol}

\usepackage{makecell}

\usepackage{color}
\usepackage{hyperref}
\hypersetup{
    colorlinks,
    citecolor=black,
    filecolor=black,
    linkcolor=black,
    urlcolor=black
}

\usepackage{amsthm}

\usepackage{tcolorbox}
\tcbuselibrary{listingsutf8}

\usepackage{ifplatform}

\usepackage{ifthen}

\usepackage{macroenvsign}


% Sections numbering

\renewcommand\thesection{\arabic{section}.}
\renewcommand\thesubsection{\alph{subsection}.}
\renewcommand\thesubsubsection{\roman{subsubsection}.}


% MISC

\newtcblisting{latexex}{%
	sharp corners,%
	left=1mm, right=1mm,%
	bottom=1mm, top=1mm,%
	colupper=red!75!blue,%
	listing side text
}

\newtcbinputlisting{\inputlatexex}[2][]{%
	listing file={#2},%
	sharp corners,%
	left=1mm, right=1mm,%
	bottom=1mm, top=1mm,%
	colupper=red!75!blue,%
	listing side text
}


\newtcblisting{latexex-flat}{%
	sharp corners,%
	left=1mm, right=1mm,%
	bottom=1mm, top=1mm,%
	colupper=red!75!blue,%
}

\newtcbinputlisting{\inputlatexexflat}[2][]{%
	listing file={#2},%
	sharp corners,%
	left=1mm, right=1mm,%
	bottom=1mm, top=1mm,%
	colupper=red!75!blue,%
}


\newtcblisting{latexex-alone}{%
	sharp corners,%
	left=1mm, right=1mm,%
	bottom=1mm, top=1mm,%
	colupper=red!75!blue,%
	listing only
}

\newtcbinputlisting{\inputlatexexalone}[2][]{%
	listing file={#2},%
	sharp corners,%
	left=1mm, right=1mm,%
	bottom=1mm, top=1mm,%
	colupper=red!75!blue,%
	listing only
}


\newcommand\inputlatexexcodeafter[1]{%
	\begin{center}
		\input{#1}
	\end{center}

	\vspace{-.5em}
	
	Le rendu précédent a été obtenu via le code suivant.
	
	\inputlatexexalone{#1}
}


\newcommand\inputlatexexcodebefore[1]{%
	\inputlatexexalone{#1}
	\vspace{-.75em}
	\begin{center}
		\textit{\footnotesize Rendu du code précédent}
		
		\medskip
		
		\input{#1}
	\end{center}
}


\newcommand\env[1]{\texttt{#1}}
\newcommand\macro[1]{\env{\textbackslash{}#1}}



\setlength{\parindent}{0cm}
\setlist{noitemsep}

\theoremstyle{definition}
\newtheorem*{remark}{Remarque}

\usepackage[raggedright]{titlesec}

\titleformat{\paragraph}[hang]{\normalfont\normalsize\bfseries}{\theparagraph}{1em}{}
\titlespacing*{\paragraph}{0pt}{3.25ex plus 1ex minus .2ex}{0.5em}


\newcommand\separation{
	\medskip
	\hfill\rule{0.5\textwidth}{0.75pt}\hfill
	\medskip
}


\newcommand\extraspace{
	\vspace{0.25em}
}


\newcommand\whyprefix[2]{%
	\textbf{\prefix{#1}}-#2%
}

\newcommand\mwhyprefix[2]{%
	\texttt{#1 = #1-#2}%
}

\newcommand\prefix[1]{%
	\texttt{#1}%
}


\newcommand\inenglish{\@ifstar{\@inenglish@star}{\@inenglish@no@star}}

\newcommand\@inenglish@star[1]{%
	\emph{\og #1 \fg}%
}

\newcommand\@inenglish@no@star[1]{%
	\@inenglish@star{#1} en anglais%
}


\newcommand\ascii{\texttt{ASCII}}


% Example
\newcounter{paraexample}[subsubsection]

\newcommand\@newexample@abstract[2]{%
	\paragraph{%
		#1%
		\if\relax\detokenize{#2}\relax\else {} -- #2\fi%
	}%
}



\newcommand\newparaexample{\@ifstar{\@newparaexample@star}{\@newparaexample@no@star}}

\newcommand\@newparaexample@no@star[1]{%
	\refstepcounter{paraexample}%
	\@newexample@abstract{Exemple \theparaexample}{#1}%
}

\newcommand\@newparaexample@star[1]{%
	\@newexample@abstract{Exemple}{#1}%
}


% Change log
\newcommand\topic{\@ifstar{\@topic@star}{\@topic@no@star}}

\newcommand\@topic@no@star[1]{%
	\textbf{\textsc{#1}.}%
}

\newcommand\@topic@star[1]{%
	\textbf{\textsc{#1} :}%
}



	\usepackage{01-operators}
\makeatother


	
\begin{document}

\section{Unions et intersections en mode ligne}

\LaTeX{} permet d'afficher sans souci $\cup_{k=1}^{n}$ mais ne propose pas $\dcup_{k=1}^{n}$.
Les macros \macro{dcap}, \macro{dcup} et \macro{dsqcup} donnent accès à ce type de fonctionnalité pour $\cap$ , $\cup$ et $\sqcup$ respectivement.
Voici des exemples d'utilisation.


% ---------------------- %


\newparaexample{Les symboles \og seuls \fg}

\begin{latexex}
$A \dcap B = C \cap D$

$A \dcup B = C \cup D$

$A \dsqcup B = C \sqcup D$
\end{latexex}


% ---------------------- %


\newparaexample{Des intersections indicées}

Ci-dessous est utilisée la macro \macro{bigcap} proposée par le package \verb+amssymb+.

\begin{latexex}
$\cap_{k=1}^{n}    A_k$ ,
$\dcap_{k=1}^{n}   B_k$ ,
$\bigcap_{k=1}^{n} C_k$ ,
$\displaystyle%
 \bigcap_{k=1}^{n} D_k$
\end{latexex}


% ---------------------- %


\newparaexample{Des unions indicées}

Ci-dessous est utilisée la macro \macro{bigcup} proposée par le package \verb+amssymb+.

\begin{latexex}
$\cup_{k=1}^{n}    A_k$ ,
$\dcup_{k=1}^{n}   B_k$ ,
$\bigcup_{k=1}^{n} C_k$ ,
$\displaystyle%
 \bigcup_{k=1}^{n} D_k$
\end{latexex}


% ---------------------- %


\newparaexample{Des unions disjointes indicées}

Ci-dessous sont utilisées les macros \macro{sqcup} et \macro{bigsqcup} proposée par le package \verb+amssymb+.

\begin{latexex}
$\sqcup_{k=1}^{n}    A_k$ ,
$\dsqcup_{k=1}^{n}   B_k$ ,
$\bigsqcup_{k=1}^{n} C_k$ ,
$\displaystyle%
 \bigsqcup_{k=1}^{n} D_k$
\end{latexex}


% ---------------------- %


\subsection{Fiches techniques}

\paragraph{Unions et intersections}

\IDmacro*{dcap}{0}

\IDmacro*{dcup}{0}

\IDmacro*{dsqcup}{0}

\end{document}
