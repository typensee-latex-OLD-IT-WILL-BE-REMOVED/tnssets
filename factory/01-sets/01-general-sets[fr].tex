\documentclass[12pt,a4paper]{article}

\makeatletter
	\usepackage[utf8]{inputenc}
\usepackage[T1]{fontenc}
\usepackage{ucs}

\usepackage[french]{babel,varioref}

\usepackage[top=2cm, bottom=2cm, left=1.5cm, right=1.5cm]{geometry}
\usepackage{enumitem}

\usepackage{multicol}

\usepackage{makecell}

\usepackage{color}
\usepackage{hyperref}
\hypersetup{
    colorlinks,
    citecolor=black,
    filecolor=black,
    linkcolor=black,
    urlcolor=black
}

\usepackage{amsthm}

\usepackage{tcolorbox}
\tcbuselibrary{listingsutf8}

\usepackage{ifplatform}

\usepackage{ifthen}

\usepackage{cbdevtool}



% MISC

\newtcblisting{latexex}{%
	sharp corners,%
	left=1mm, right=1mm,%
	bottom=1mm, top=1mm,%
	colupper=red!75!blue,%
	listing side text
}

\newtcbinputlisting{\inputlatexex}[2][]{%
	listing file={#2},%
	sharp corners,%
	left=1mm, right=1mm,%
	bottom=1mm, top=1mm,%
	colupper=red!75!blue,%
	listing side text
}


\newtcblisting{latexex-flat}{%
	sharp corners,%
	left=1mm, right=1mm,%
	bottom=1mm, top=1mm,%
	colupper=red!75!blue,%
}

\newtcbinputlisting{\inputlatexexflat}[2][]{%
	listing file={#2},%
	sharp corners,%
	left=1mm, right=1mm,%
	bottom=1mm, top=1mm,%
	colupper=red!75!blue,%
}


\newtcblisting{latexex-alone}{%
	sharp corners,%
	left=1mm, right=1mm,%
	bottom=1mm, top=1mm,%
	colupper=red!75!blue,%
	listing only
}

\newtcbinputlisting{\inputlatexexalone}[2][]{%
	listing file={#2},%
	sharp corners,%
	left=1mm, right=1mm,%
	bottom=1mm, top=1mm,%
	colupper=red!75!blue,%
	listing only
}


\newcommand\inputlatexexcodeafter[1]{%
	\begin{center}
		\input{#1}
	\end{center}

	\vspace{-.5em}
	
	Le rendu précédent a été obtenu via le code suivant.
	
	\inputlatexexalone{#1}
}


\newcommand\inputlatexexcodebefore[1]{%
	\inputlatexexalone{#1}
	\vspace{-.75em}
	\begin{center}
		\textit{\footnotesize Rendu du code précédent}
		
		\medskip
		
		\input{#1}
	\end{center}
}


\newcommand\env[1]{\texttt{#1}}
\newcommand\macro[1]{\env{\textbackslash{}#1}}



\setlength{\parindent}{0cm}
\setlist{noitemsep}

\theoremstyle{definition}
\newtheorem*{remark}{Remarque}

\usepackage[raggedright]{titlesec}

\titleformat{\paragraph}[hang]{\normalfont\normalsize\bfseries}{\theparagraph}{1em}{}
\titlespacing*{\paragraph}{0pt}{3.25ex plus 1ex minus .2ex}{0.5em}


\newcommand\separation{
	\medskip
	\hfill\rule{0.5\textwidth}{0.75pt}\hfill
	\medskip
}


\newcommand\extraspace{
	\vspace{0.25em}
}


\newcommand\whyprefix[2]{%
	\textbf{\prefix{#1}}-#2%
}

\newcommand\mwhyprefix[2]{%
	\texttt{#1 = #1-#2}%
}

\newcommand\prefix[1]{%
	\texttt{#1}%
}


\newcommand\inenglish{\@ifstar{\@inenglish@star}{\@inenglish@no@star}}

\newcommand\@inenglish@star[1]{%
	\emph{\og #1 \fg}%
}

\newcommand\@inenglish@no@star[1]{%
	\@inenglish@star{#1} en anglais%
}


\newcommand\ascii{\texttt{ASCII}}


% Example
\newcounter{paraexample}[subsubsection]

\newcommand\@newexample@abstract[2]{%
	\paragraph{%
		#1%
		\if\relax\detokenize{#2}\relax\else {} -- #2\fi%
	}%
}



\newcommand\newparaexample{\@ifstar{\@newparaexample@star}{\@newparaexample@no@star}}

\newcommand\@newparaexample@no@star[1]{%
	\refstepcounter{paraexample}%
	\@newexample@abstract{Exemple \theparaexample}{#1}%
}

\newcommand\@newparaexample@star[1]{%
	\@newexample@abstract{Exemple}{#1}%
}


% Change log
\newcommand\topic{\@ifstar{\@topic@star}{\@topic@no@star}}

\newcommand\@topic@no@star[1]{%
	\textbf{\textsc{#1}.}%
}

\newcommand\@topic@star[1]{%
	\textbf{\textsc{#1} :}%
}



	\usepackage{01-general-sets}
\makeatother


\usepackage{relsize}



\begin{document}

\section{Ensembles}

\subsection{Différents types d'ensembles}

\subsubsection{Ensembles versus accolades}

\newparaexample{}

\begin{latexex}
$\setgene{1 ; 3 ; 5}$ .
\end{latexex}


% ---------------------- %


\newparaexample{}

Dans l'exemple suivant on utilise l'option \prefix{sb} pour \whyprefix{s}{mall} \whyprefix{b}{races} soit \inenglish{petites accolades}.

\begin{latexex}
$\setgene {\dfrac{1}{3} ; \dfrac{5}{7} ; 
           \dfrac{9}{11}}$

$\setgene*{\dfrac{1}{3} ; \dfrac{5}{7} ;
           \dfrac{9}{11}}$
\end{latexex}


% ---------------------- %


\subsubsection{Fiches techniques}

\paragraph{Ensembles versus accolades}

\IDmacro{setgene}{1}{1}

\IDoption{} la valeur par défaut est \verb+b+.  Voici les différentes valeurs possibles.
\begin{enumerate}
	\item \verb+b+ : on utilise des accolades extensibles.

	\item \verb+sb+ : on utilise des accolades non extensibles.
\end{enumerate}

\IDarg{} la définition de l'ensemble.

\IDarg{} la définition de l'ensemble.


% ---------------------- %


\subsubsection{Ensembles pour la géométrie} \label{set-geo}

\newparaexample{}

\begin{latexex}
$\setgeo{C}$ ,
$\setgeo{D}$ ou
$\setgeo{d}$
\end{latexex}

\begin{remark}
	Pour le moment, il n'est pas possible de taper \verb+$\setgeo{ABC}$+ avec plusieurs lettres.
\end{remark}


% ---------------------- %


\newparaexample{Avec des indices}

\begin{latexex}
$\setgeo*{C}{1}$ ou
$\setgeo*{C}{2}$
\end{latexex}


% ---------------------- %


\subsubsection{Fiches techniques}

\paragraph{Ensembles pour la géométrie}

\IDmacro*{setgeo}{1}

\IDarg{} un seul caractère \ascii{} indiquant un ensemble géométrique.


\separation


\IDmacro*{setgeo*}{2}

\IDarg{1} un seul caractère \ascii{} indiquant $\setgeo{U}$ dans le nom $\setgeo*{U}{d}$ d'un ensemble géométrique.

\IDarg{2} un texte donnant $d$ dans le nom $\setgeo*{U}{d}$ d'un ensemble géométrique.


% ---------------------- %


\subsubsection{Ensembles probabilistes}

\newparaexample{}

\begin{latexex}
$\setproba{E}$ ou
$\setproba{G}$
\end{latexex}

\begin{remark}
	Pour le moment, il n'est pas possible de taper \verb+$\setproba{ABC}$+ avec plusieurs lettres.
\end{remark}


% ---------------------- %


\newparaexample{Avec des indices}

\begin{latexex}
$\setproba*{E}{1}$ ou
$\setproba*{E}{2}$
\end{latexex}


% ---------------------- %


\subsubsection{Fiches techniques}

\paragraph{Ensembles probabilistes}

\IDmacro*{setproba}{1}

\IDarg{} un seul caractère \ascii{} majuscule indiquant un ensemble probabiliste.


\separation


\IDmacro*{setproba*}{2}

\IDarg{1} un seul caractère \ascii{} majuscule indiquant $\setproba{U}$ dans le nom $\setproba*{U}{d}$ d'un ensemble probabiliste.

\IDarg{2} un texte donnant $d$ dans le nom $\setproba*{U}{d}$ d'un ensemble probabiliste.


% ---------------------- %


\subsubsection{Ensembles pour l'algèbre générale}

\newparaexample{}

\begin{latexex}
$\setalge{A}$ ,
$\setalge{K}$ ,
$\setalge{h}$ ou
$\setalge{k}$
\end{latexex}

\begin{remark}
	Pour le moment, il n'est pas possible de taper \verb+$\setalge{ABC}$+ avec plusieurs lettres.
\end{remark}


% ---------------------- %


\newparaexample{Avec des indices}

\begin{latexex}
$\setalge*{k}{1}$ ou $\setalge*{k}{2}$
\end{latexex}


% ---------------------- %


\subsubsection{Fiches techniques}

\paragraph{Ensembles pour l'algèbre générale}

\IDmacro*{setalge}{1}

\IDarg{} soit l'une des lettres  \texttt{h} et \texttt{k}, soit un seul caractère \ascii{} majuscule indiquant un ensemble de type anneau ou corps.


\separation


\IDmacro*{setalge*}{2}

\IDarg{1} un seul caractère \ascii{} indiquant $\setalge{U}$ dans le nom $\setalge*{U}{d}$ d'un ensemble de type anneau ou corps.

\IDarg{2} un texte donnant $d$ dans le nom $\setalge*{U}{d}$ d'un ensemble de type anneau ou corps.


% ---------------------- %


\subsection{Ensembles classiques en mathématiques et en informatique théorique} 

\subsubsection{La liste complète}

Dans l'exemple suivant,
$\PP$ désigne l'ensemble des nombres premiers,
$\HH$ celui des quaternions,
$\OO$ celui des octonions et
$\FF$ un ensemble de nombres flottants \emph{(notation à préciser suivant le contexte)}.

\begin{latexex}
$\nullset$

$\NN$ , $\ZZ$ , $\PP$

$\DD$ , $\QQ$ , $\RR$ , $\CC$

$\HH$ , $\OO$

$\FF$
\end{latexex}


% ---------------------- %


\subsubsection{Ensembles classiques suffixés}

L'ensemble $\RR$ nous permet de voir tous les cas possibles. 

\begin{latexex}
$\RRn$ , $\RRp$ , $\RRs$ 

$\RRsn$ , $\RRsp$
\end{latexex}


Nous avons utilisé les suffixes \prefix{n} pour \whyprefix{n}{égatif}, \prefix{p} pour \whyprefix{p}{ositif} et \prefix{s} pour \whyprefix{s}{tar} soit \inenglish{étoile}. Il y a aussi les suffixes composites \prefix{sn} et \prefix{sp}.

\medskip

Notez qu'il est interdit d'utiliser \verb+$\CCn$+ pour $\setspecial{\CC}{n}$ car l'ensemble $\CC$ ne possède pas de structure ordonnée standard. Jetez un oeil à la section suivante pour apprendre à taper $\setspecial{\CC}{n}$ si vous en avez besoin. L'interdiction est ici purement sémantique !

\medskip

\begin{remark}
	La table \ref{tnssets-table:suffixes-sets} \vpageref{tnssets-table:suffixes-sets} montre les associations autorisées entre ensembles classiques et suffixes.
\end{remark}

% == Table of suffixes - START == %

\begin{table}[h]
    \caption{Suffixes}
    \begin{center}
        \begin{tabular}{c|c|c|c|c|c}
              & \verb+n+ & \verb+p+ & \verb+s+ & \verb+sn+ & \verb+sp+ \\
            \hline \makecell{\macro{NN}} &          &          & $\times$ &          &          \\
            \hline \makecell{\macro{PP}} &          &          &          &          &          \\
            \hline \makecell{\macro{ZZ}\\\macro{DD}\\\macro{QQ}\\\macro{RR}} & $\times$ & $\times$ & $\times$ & $\times$ & $\times$ \\
            \hline \makecell{\macro{CC}\\\macro{HH}\\\macro{OO}} &          &          & $\times$ &          &          \\
            \hline \makecell{\macro{FF}} & $\times$ & $\times$ & $\times$ & $\times$ & $\times$ \\
        \end{tabular}
    \end{center}
    \label{tnssets-table:suffixes-sets}
\end{table}

% == Table of suffixes - END == %


% ---------------------- %


\subsubsection{Fiches techniques}

\paragraph{Ensembles classiques}

% == Docs for classical sets - START == %
\foreach \k in {NN, NNs}{

    \IDmacro*{\k}{0}
}
                
\separation

\foreach \k in {PP}{

    \IDmacro*{\k}{0}
}
                
\separation

\foreach \k in {ZZ, ZZn, ZZp, ZZs, ZZsn, ZZsp}{

    \IDmacro*{\k}{0}
}
                
\separation

\foreach \k in {DD, DDn, DDp, DDs, DDsn, DDsp}{

    \IDmacro*{\k}{0}
}
                
\separation

\foreach \k in {QQ, QQn, QQp, QQs, QQsn, QQsp}{

    \IDmacro*{\k}{0}
}
                
\separation

\foreach \k in {RR, RRn, RRp, RRs, RRsn, RRsp}{

    \IDmacro*{\k}{0}
}
                
\separation

\foreach \k in {CC, CCs}{

    \IDmacro*{\k}{0}
}
                
\separation

\foreach \k in {HH, HHs}{

    \IDmacro*{\k}{0}
}
                
\separation

\foreach \k in {OO, OOs}{

    \IDmacro*{\k}{0}
}
                % == Docs for classical sets - END == %


% ---------------------- %


\subsection{Des suffixes à la carte}

\paragraph{Exemple d'utilisation}

Dans cet exemple, il faut savoir que le 2\ieme{} argument ne peut prendre que les valeurs \prefix{n}, \prefix{p}, \prefix{s}, \prefix{sn} ou \prefix{sp}.

\begin{latexex}
$\setspecial{\CC}{n}$ ,
$\setspecial{\HH}{sp}$ ou
$\setspecial*{\setproba{P}}{n}$
\end{latexex}


% ---------------------- %


\subsubsection{Fiches techniques}

\paragraph{Ensembles -- Des suffixes à la carte}

\IDmacro*{setspecial}{2}

\IDmacro*{setspecial*}{2}

\IDarg{1} l'ensemble à "suffixer".

\IDarg{2} l'un des suffixes \prefix{n}, \prefix{p}, \prefix{s}, \prefix{sn} ou \prefix{sp}.

\end{document}
