\documentclass[12pt,a4paper]{article}

\makeatletter
    \usepackage[utf8]{inputenc}
\usepackage[T1]{fontenc}
\usepackage{ucs}

\usepackage[french]{babel,varioref}

\usepackage[top=2cm, bottom=2cm, left=1.5cm, right=1.5cm]{geometry}
\usepackage{enumitem}

\usepackage{multicol}

\usepackage{makecell}

\usepackage{color}
\usepackage{hyperref}
\hypersetup{
    colorlinks,
    citecolor=black,
    filecolor=black,
    linkcolor=black,
    urlcolor=black
}

\usepackage{amsthm}

\usepackage{tcolorbox}
\tcbuselibrary{listingsutf8}

\usepackage{ifplatform}

\usepackage{ifthen}

\usepackage{cbdevtool}



% MISC

\newtcblisting{latexex}{%
	sharp corners,%
	left=1mm, right=1mm,%
	bottom=1mm, top=1mm,%
	colupper=red!75!blue,%
	listing side text
}

\newtcbinputlisting{\inputlatexex}[2][]{%
	listing file={#2},%
	sharp corners,%
	left=1mm, right=1mm,%
	bottom=1mm, top=1mm,%
	colupper=red!75!blue,%
	listing side text
}


\newtcblisting{latexex-flat}{%
	sharp corners,%
	left=1mm, right=1mm,%
	bottom=1mm, top=1mm,%
	colupper=red!75!blue,%
}

\newtcbinputlisting{\inputlatexexflat}[2][]{%
	listing file={#2},%
	sharp corners,%
	left=1mm, right=1mm,%
	bottom=1mm, top=1mm,%
	colupper=red!75!blue,%
}


\newtcblisting{latexex-alone}{%
	sharp corners,%
	left=1mm, right=1mm,%
	bottom=1mm, top=1mm,%
	colupper=red!75!blue,%
	listing only
}

\newtcbinputlisting{\inputlatexexalone}[2][]{%
	listing file={#2},%
	sharp corners,%
	left=1mm, right=1mm,%
	bottom=1mm, top=1mm,%
	colupper=red!75!blue,%
	listing only
}


\newcommand\inputlatexexcodeafter[1]{%
	\begin{center}
		\input{#1}
	\end{center}

	\vspace{-.5em}
	
	Le rendu précédent a été obtenu via le code suivant.
	
	\inputlatexexalone{#1}
}


\newcommand\inputlatexexcodebefore[1]{%
	\inputlatexexalone{#1}
	\vspace{-.75em}
	\begin{center}
		\textit{\footnotesize Rendu du code précédent}
		
		\medskip
		
		\input{#1}
	\end{center}
}


\newcommand\env[1]{\texttt{#1}}
\newcommand\macro[1]{\env{\textbackslash{}#1}}



\setlength{\parindent}{0cm}
\setlist{noitemsep}

\theoremstyle{definition}
\newtheorem*{remark}{Remarque}

\usepackage[raggedright]{titlesec}

\titleformat{\paragraph}[hang]{\normalfont\normalsize\bfseries}{\theparagraph}{1em}{}
\titlespacing*{\paragraph}{0pt}{3.25ex plus 1ex minus .2ex}{0.5em}


\newcommand\separation{
	\medskip
	\hfill\rule{0.5\textwidth}{0.75pt}\hfill
	\medskip
}


\newcommand\extraspace{
	\vspace{0.25em}
}


\newcommand\whyprefix[2]{%
	\textbf{\prefix{#1}}-#2%
}

\newcommand\mwhyprefix[2]{%
	\texttt{#1 = #1-#2}%
}

\newcommand\prefix[1]{%
	\texttt{#1}%
}


\newcommand\inenglish{\@ifstar{\@inenglish@star}{\@inenglish@no@star}}

\newcommand\@inenglish@star[1]{%
	\emph{\og #1 \fg}%
}

\newcommand\@inenglish@no@star[1]{%
	\@inenglish@star{#1} en anglais%
}


\newcommand\ascii{\texttt{ASCII}}


% Example
\newcounter{paraexample}[subsubsection]

\newcommand\@newexample@abstract[2]{%
	\paragraph{%
		#1%
		\if\relax\detokenize{#2}\relax\else {} -- #2\fi%
	}%
}



\newcommand\newparaexample{\@ifstar{\@newparaexample@star}{\@newparaexample@no@star}}

\newcommand\@newparaexample@no@star[1]{%
	\refstepcounter{paraexample}%
	\@newexample@abstract{Exemple \theparaexample}{#1}%
}

\newcommand\@newparaexample@star[1]{%
	\@newexample@abstract{Exemple}{#1}%
}


% Change log
\newcommand\topic{\@ifstar{\@topic@star}{\@topic@no@star}}

\newcommand\@topic@no@star[1]{%
	\textbf{\textsc{#1}.}%
}

\newcommand\@topic@star[1]{%
	\textbf{\textsc{#1} :}%
}



    \usepackage{02-applicatons}
\makeatother

    
\begin{document}

%\section{Applications}

\subsection{Application totale, partielle, injective, surjective et/ou bijective}

Voici des symboles qui, bien que très techniques, facilitent la rédaction de documents à propos des applications totales ou partielles
\footnote{
	$a: E \to F$ est une application totale si $\forall x \in E$, $\exists\,! y \in F$ tel que $y = a(x)$. 
	Plus généralement, $f: E \to F$ est une application partielle si $\forall x \in E$, $\exists_{\leq 1} y \in F$ tel que $y = f(x)$, autrement dit soit $f(x)$ existe dans $F$, soit $f$ n'est pas définie en $x$.
}
\emph{(on parle aussi d'applications, sans qualificatif, et de fonctions)}.


% ---------------------- %


\newparaexample{Applications totales}

\begin{latexex-flat}
$f: A \to B$ est une application totale, c'est à dire définie sur $A$ tout entier.

$i: C \onetoone D$ est une application totale injective.

$s: E \onto F$ est une application totale surjective.

$b: G \biject H$ est une application totale bijective.
\end{latexex-flat}


% ---------------------- %


\newparaexample{Applications partielles}

\begin{latexex-flat}
$f: A \pto B$ est une application partielle, c'est à dire définie sur
un sous-ensemble de $A$.

$i: C \ponetoone D$ est une application partielle injective.

$s: E \ponto F$ est une application partielle surjective.

$b: G \pbiject H$ est une application partielle bijective.
\end{latexex-flat}


% ---------------------- %


\subsection{Fiches techniques}

\vspace{-.75em}
\begin{multicols}{2}
    \IDope{to}
    
    \IDope{onetoone}
    
    \IDope{onto}
    
    \IDope{biject}
    
    \IDope{pto}
    
    \IDope{ponetoone}
    
    \IDope{ponto}
    
    \IDope{pbiject}
\end{multicols}
\end{document}
